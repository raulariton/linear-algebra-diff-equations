\documentclass[a4paper,12pt]{article}
\usepackage[english]{babel}
\usepackage[utf8]{inputenc}
\usepackage[T1]{fontenc}

\usepackage{graphicx}

\usepackage{lmodern}

% math symbols and fonts
\usepackage{amsmath}
\usepackage{amssymb}
\usepackage{mathrsfs}

% d for derivatives
\usepackage{fixdif}

% SI units
\usepackage{siunitx}

% framed environment, might use in problem introduction
\usepackage{framed}

% bibtex setup
\usepackage[backend=biber,
            style=ieee,
            ]{biblatex}
\addbibresource{bibliography.bib}

% context-sensitive quotes package (biblatex suggests using it, idk what it does exactly)
\usepackage{csquotes}

% hyperref package; convention to always add as last package
\usepackage{hyperref}

% adding text spacing in math mode
\renewcommand*{\textnormal}[1]{\text{ #1 }}

% removing highlighting of hyperref links
\hypersetup{pdfborder={0 0 0}}

% remove indentation
\setlength{\parindent}{0pt}

\newcommand\numberthis{\addtocounter{equation}{1}\tag{\theequation}}

\begin{document}

\begin{titlepage}
	\vspace*{-3cm}
	\centering
		\includegraphics[width=2cm]{assets/uvt_logo_en.png}

	{\scshape\LARGE West University of Timișoara\par}

	\vspace{1cm}

	{\large Faculty of Computer Science}

	\vspace{.1\textheight}
	{ \LARGE \scshape Project 1: \\ Snowplow Problem \par}

		\vspace{.1\linewidth}

			{\emph{Authors:} \\
			Maria-Miruna \textsc{Mesărășoiu} (\textit{Group 1}) \\
            Silviu-Ștefan \textsc{Mitrea} (\textit{Group 1}) \\
            Raul-Andrei \textsc{Ariton} (\textit{Group 2})    
            }

			\vfill

			\rule{.4\textwidth}{.4pt}

	{\large \today\par
	Academic Year 2025-2026\par}
\end{titlepage}

\tableofcontents \newpage

\part{The Snowplow Problem}
\section{Introduction to the problem}
    Firstly, we shall introduce the problem, as described in \cite{nagle2012fundamentals}. Note that for practicality, we have decided to use kilometers instead of miles.

    \begin{framed}
        \noindent One morning it began to snow very hard and continued snowing steadily throughout the day. A snowplow set out at 9:00 AM to clear a road, clearing 2 \unit{\kilo\meter} by 11:00 AM. and an additional kilometer by 1:00 PM. At what time did it start snowing?
    \end{framed}
    % image, tikZ plot ? :)
    
    The source also mentions the following assumptions:
    \begin{itemize}
        \item It is snowing at a constant rate.
        \item The rate at which the snowplow can clear a road is inversely proportional to the height (or depth) of the snow.
    \end{itemize}

\section{Solution}
    We shall now begin the attempt at solving the problem. 

    \noindent Let us consider the following notations:
    \begin{itemize}
        \item $t$ - time since 9:00 AM (hours)
        \item $x(t)$ - the distance that the snowplow has travelled ($\unit{\kilo\meter}$) at time $t$
        \item $h(t)$ - the height of the snow ($\unit{\centi\meter}$) at time $t$
    \end{itemize}
    Additionally:
    \begin{itemize}
        \item b - the number of hours before 9:00 AM when it started snowing
        \item r - rate of snowfall ($\unit{\centi\meter / h}$)
    \end{itemize}

    Since it is snowing at a constant rate, the height of the snow $h$ is increasing at a constant rate. Mathematically this means that the derivative of $h$ is constant, and equal to $r$, $r \in \mathbb{R}$.

    \begin{align*}
        h'(t) = \frac{\d h}{\d t} = r
    \end{align*}

    Thus $h$ is a linear function with slope $r$ and with the equation
    \begin{align*}
        h(t) = r t+ C, \quad C \in \mathbb{R}
    \end{align*}
    That's a lot of letters and unknowns. Let us attempt to substitute the constant $C$ in the equation of $h(t)$ with something more relevant to our problem.
    We consider $t = 0$ the moment that the snowplow sets out to clear the road. Similarly, $t = 0-b$ is the moment that it starts snowing, and the height of the snow is equal to $0$.
    \begin{align*}
        &h(0-b) = r (0-b) + C \\
        &0 = -rb + C \\
        &\boxed{C = rb}
    \end{align*}
    And now we can substitute $C$ in the equation of $h$:
    \begin{align*}
        h(t) = rt + rb \\
        h(t) = r(t+b)
    \end{align*}
    We know that the rate at which the snowplow moves (clears a road) is inversely proportional to the height of the snow:
    \begin{align*}
        x'(t) = \frac{\d x}{\d t} \propto \frac{1}{h(t)}
    \end{align*}
    Then the product of the two is constant:
    \begin{align*}
        x'(t) h(t) &= m, \quad m \in \mathbb{R} \\
        \iff x'(t) &= \frac{m}{h(t)} \\
        \iff x'(t) &= \frac{m}{r(t+b)}
    \end{align*}
    Since $m$ and $r$ are constants, we can rewrite the equation as follows, separating the constants from the variables and unknowns:
    \begin{equation*}
        x'(t) = \frac{m}{r} \frac{1}{(t+b)}
    \end{equation*}

    % ^ this is a differential equation

    Since we have information about the kilometers the snowplow clears after a known number of hours, let us attempt to find the equation of $x(t)$. For this, we will integrate $x'(t)$:
    \begin{align*}
        x(t) &= \int x'(t) \d t \\ 
        &= \int \frac{m}{r} \frac{1}{(t+b)} \d t \\
        &= \frac{m}{r} \int \frac{1}{(t+b)} \d t \\
        &= \frac{m}{r} \ln(t+b) + C_2, \quad C_2 \in \mathbb{R}
    \end{align*}
    Note that the product $\frac{m}{r}C_2$ is just a product of constants and thus unnecessary to write.
    Now using the equation for $x(t)$, we substitute using known information:
    \begin{itemize}
        \item The snowplow cleared $2 \unit{\kilo\meter}$ between 9:00 AM and 11:00 AM (the span of two hours):
        \begin{align*}
            x(2) - x(0) &= \left.\frac{m}{r}\ln(t+b)\right|^2_0 \\
            2 &= \left(\frac{m}{r} \ln(2+b)\right) - \left(\frac{m}{r}\ln(0+b)\right) \\
            2 &= \frac{m}{r} \left(\ln(2+b) - \ln(0+b)\right) \\
            &\boxed{2 = \frac{m}{r} \ln\left(\frac{2+b}{b}\right)} \numberthis \label{eq:firstunknown}
        \end{align*}
        \item The snowplow cleared $1 \unit{\kilo\meter}$ between 11:00 AM (2 hours after it set out) and 1:00 PM (4 hours after it set out):
        \begin{align*}
            x(4) - x(2) &= \left.\frac{m}{r}\ln(t+b)\right|^4_2 \\
            1 &= \left(\frac{m}{r} \ln(4+b)\right) - \left(\frac{m}{r}\ln(2+b)\right) \\
            1 &= \frac{m}{r} \left(\ln(4+b) - \ln(2+b)\right) \\
            &\boxed{1 = \frac{m}{r} \ln\left(\frac{4+b}{2+b}\right)} \numberthis \label{eq:secondunknown}
        \end{align*}
    \end{itemize}

    We now have two equations with $b$ as our unknown. We shall now try to combine the two and solve for $b$:

    If we multiply \eqref{eq:secondunknown} by 2:
    \begin{align*}
        &\left.1 = \frac{m}{r} \ln\left(\frac{4+b}{2+b}\right)\right|\times 2 \\
        &2 = \frac{m}{r} 2 \ln\left(\frac{4+b}{2+b}\right) \\
        2 &= \frac{m}{r} \ln\left(\frac{4+b}{2+b}\right)^2 \\
        &\boxed{\frac{2r}{m} = \ln\left(\frac{4+b}{2+b}\right)^2} \label{eq:thirdunknown} \numberthis
    \end{align*}

    Then if we solve for the $\ln$ in \eqref{eq:firstunknown}:
    \begin{align*}
        &\frac{m}{r} \ln\left(\frac{2+b}{b}\right) = 2 \\
        &\boxed{\frac{2r}{m} = \ln\left(\frac{2+b}{b}\right)} \numberthis \label{eq:fourthunknown}
    \end{align*}

    Then, equating the right hand side members of \eqref{eq:thirdunknown} and \eqref{eq:fourthunknown}:
    \begin{align*}
        \ln\left(\frac{2+b}{b}\right) &= \ln\left(\frac{4+b}{2+b}\right)^2 \\
        \frac{2+b}{b} &= \left(\frac{4+b}{2+b}\right)^2 \\
        \frac{2+b}{b} &= \frac{(4+b)^2}{(2+b)^2} \\
        (2+b)^3 &= b(4+b)^2 \\
        b^3 + 6b^2 + 12b + 8 &= b (b^2 + 8b + 16) \\
        \cancel{b^3} + 6b^2 + 12b + 8 &= \cancel{b^3} + 8b^2 + 16b \\
         2b^2 +  4b - 8 &= 0 \ \bigg| \div 2 \\ 
         b^2 + 2b - 4 &= 0 \\
         \overset{\text{quadratic formula}}{\iff} b_1 = \sqrt{5} - 1 \approx 1.236 &\qquad b_2 = -\sqrt 5 - 1 \approx - 3.236
    \end{align*}

    Since $b$ is the number of hours before 9:00 AM, the moment at which the snowplow set out to clear the snow, $b$ must be a positive number so $b >0$ and the second solution $b_2$ is rejected.

    In conclusion, it started snowing $1.236$ hours before 9:00 AM. If we want to be specific:
    \begin{align*}
        0.236 \textnormal{hours is} 0.236 \cdot 60 = 14.16 \textnormal{minutes} \\
        0.16 \textnormal{minutes is} 0.16 \cdot 60 = 9.6 \textnormal{seconds}
    \end{align*}
    so $1.236$ hours equate to $1$ hour, $14$ minutes and $9.6$ seconds
    \[
        9{:}00{:}00 - 1{:}14{:}9.6 = 7{:}45{:}50.4 \textnormal{AM}
    \]
   It started snowing at $7{:}45{:}50.4$ AM.

\part{Two Snowplows}
\section{The Problem}
\begin{framed}
    \noindent One morning it began to snow very hard and continued snowing steadily throughout the day. A snowplow set out at 9:00 AM to clear a road, clearing 2 \unit{\kilo\meter} by 11:00 AM. and an additional kilometer by 1:00 PM. At what time did it start snowing?
\end{framed}
\section{The Solution}

    
    

\newpage \printbibliography

\end{document}