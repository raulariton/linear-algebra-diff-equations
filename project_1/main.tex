\documentclass[a4paper,12pt]{article}
\usepackage[english]{babel}
\usepackage[utf8]{inputenc}
\usepackage[T1]{fontenc}
\usepackage{listings}
\usepackage{xcolor}

\lstset{
  language=Python,
  basicstyle=\ttfamily\small,
  keywordstyle=\color{blue},
  stringstyle=\color{orange},
  commentstyle=\color{gray},
  numbers=left,
  numberstyle=\tiny\color{gray},
  frame=single,
  breaklines=true,
}

\usepackage{graphicx}

\usepackage{lmodern}

% math symbols and fonts
\usepackage{amsmath}
\usepackage{amssymb}
\usepackage{mathrsfs}

% d for derivatives
\usepackage{fixdif}

% SI units
\usepackage{siunitx}

% framed environment, might use in problem introduction
\usepackage{framed}

% bibtex setup
\usepackage[backend=biber,
            style=ieee,
            ]{biblatex}
\addbibresource{bibliography.bib}

\usepackage{cancel}

% context-sensitive quotes package (biblatex suggests using it, idk what it does exactly)
\usepackage{csquotes}

% hyperref package; convention to always add as last package
\usepackage{hyperref}

% adding text spacing in math mode
\renewcommand*{\textnormal}[1]{\text{ #1 }}

% removing highlighting of hyperref links
\hypersetup{pdfborder={0 0 0}}

% remove indentation
\setlength{\parindent}{0pt}

\newcommand\numberthis{\addtocounter{equation}{1}\tag{\theequation}}

\begin{document}

\begin{titlepage}
	\vspace*{-3cm}
	\centering
		\includegraphics[width=2cm]{assets/uvt_logo_en.png}

	{\scshape\LARGE West University of Timișoara\par}

	\vspace{1cm}

	{\large Faculty of Computer Science}

	\vspace{.1\textheight}
	{ \LARGE \scshape Project 1: \\ Snowplow Problem \par}

		\vspace{.1\linewidth}

			{\emph{Authors:} \\
			Maria-Miruna \textsc{Mesaroșiu} (\textit{Group 1}) \\
            Silviu-Ștefan \textsc{Mitrea} (\textit{Group 1}) \\
            Raul-Andrei \textsc{Ariton} (\textit{Group 2})    
            }

			\vfill

			\rule{.4\textwidth}{.4pt}

	{\large \today\par
	Academic Year 2025-2026\par}
\end{titlepage}

\tableofcontents \newpage

\part{The Snowplow Problem} \label{firstpart}
\section{Problem}
    Firstly, we shall introduce the problem, as described in \cite[pp.~84--85]{Kent_Nagle2018-ig}. Note that for practicality, we have decided to use kilometers instead of miles.

    \begin{framed}
        \noindent One morning it began to snow very hard and continued snowing steadily throughout the day. A snowplow set out at 9:00 AM to clear a road, clearing 2 \unit{\kilo\meter} by 11:00 AM and an additional kilometer by 1:00 PM. At what time did it start snowing?
    \end{framed}
    % image, tikZ plot ? :)
    
    The source also mentions the following assumptions:
    \begin{itemize}
        \item It is snowing at a constant rate.
        \item The rate at which the snowplow can clear a road is inversely proportional to the height (or depth) of the snow.
    \end{itemize}

\section{Solution}
    We shall now begin the attempt at solving the problem. 

    \noindent Let us consider the following notation:
    \begin{itemize}
        \item $t$ - time since it started snowing (hours)
        \item $x(t)$ - the distance that the snowplow has traveled ($\unit{\kilo\meter}$) at time $t$
        \item $h(t)$ - the height of the snow ($\unit{\centi\meter}$) at time $t$
    \end{itemize}
    Additionally:
    \begin{itemize}
        \item b - the number of hours that passed from when it started snowing until 9:00 AM
        \item r - rate of snowfall ($\unit{\centi\meter / h}$)
    \end{itemize}

    Since it is snowing at a constant rate, the height of the snow $h$ is increasing at a constant rate $r$, proportional to the time that passed. Mathematically this means that the derivative of $h$ is constant, and equal to $r$, $r \in \mathbb{R}$.
    
    \begin{align*}
        h'(t) = \frac{\d h}{\d t} = r
    \end{align*}

    Thus $h$ is a linear function with slope $r$ and with the equation
    \begin{align*}
        h(t) = r t+ C, \quad C \in \mathbb{R}
    \end{align*}
    That's a lot of letters and unknowns. Let us attempt to substitute the constant $C$ in the equation of $h(t)$ with something more relevant to our problem.
    We consider $t = 0$ the moment that it starts snowing, and the height of the snow is equal to $0$. Similarly, $t = 0+b$ is the moment that the snowplow sets out to clear the road.
    
    \begin{align*}
        &h(0) = r \cdot 0 + C \\
        &0 = 0 + C \\
        &\boxed{C = 0}
    \end{align*}
    And now we can substitute $C$ in the equation of $h$:
    \begin{align*}
        h(t) = rt 
    \end{align*}
    Since the rate at which the snowplow moves (clears a road) is inversely proportional to the height of the snow, their product is constant (called a \emph{proportionality constant}) and we shall denote it with $m$:
    \begin{align*}
        x'(t) h(t) &= m, \quad m \in \mathbb{R} \\
        \iff x'(t) &= \frac{m}{h(t)} \\
        \iff x'(t) &= \frac{m}{rt}
    \end{align*}
    Since $m$ and $r$ are constants, we can rewrite the equation as follows, separating the constants from the variables and unknowns:
    \begin{equation*}
        x'(t) = \frac{m}{r} \frac{1}{t}
    \end{equation*}

    % ^ this is a differential equation

    Since we have information about the kilometers the snowplow clears after a known number of hours, let us attempt to find the equation of $x(t)$. For this, we will integrate $x'(t)$:
    \begin{align*}
        x(t) &= \int x'(t) \d t \\ 
        &= \int \frac{m}{r} \frac{1}{t} \d t \\
        &= \frac{m}{r} \int \frac{1}{t} \d t \\
        &= \frac{m}{r} \ln(t) + C_1, \quad C_1 \in \mathbb{R}
    \end{align*}
    
    Note that the product $\frac{m}{r}C_1$ is just a product of constants and thus unnecessary to write.
    Now using the equation for $x(t)$, we substitute using known information:
    \begin{itemize}
        \item The snowplow cleared $2 \unit{\kilo\meter}$ between 9:00 AM ($t_1$) and 11:00 AM ($t_2$) (the span of two hours):
        \begin{align*}
            x(t_2) - x(t_1) &= \left.\frac{m}{r}\ln(t)\right|_{t_1}^{t_2}\\
            2 &= \left(\frac{m}{r} \ln(t_2)\right) - \left(\frac{m}{r}\ln(t_1)\right) \\
            2 &= \frac{m}{r} \left(\ln(t_2) - \ln(t_1)\right) \\
            2 &= \frac{m}{r} \ln\left(\frac{t_2}
            {t_1}\right) \\
            &\boxed{2 = \frac{m}{r} \ln\left(\frac{b+2}{b}\right)} \numberthis \label{eq:firstunknown}
        \end{align*}
        
        \item The snowplow cleared $1 \unit{\kilo\meter}$ between 11:00 AM ($t_2$) and 1:00 PM ($t_3$):
        \begin{align*}
            x(t_3) - x(t_2) &= \left.\frac{m}{r}\ln(t)\right|_{t_2}^{t_3} \\
            1 &= \left(\frac{m}{r} \ln(t_3)\right) - \left(\frac{m}{r}\ln(t_2)\right) \\
            1 &= \frac{m}{r} \left(\ln(t_3) - \ln(t_2)\right) \\
            1 &= \frac{m}{r} \ln\left(\frac{t_3}
            {t_2}\right) \\
            &\boxed{1 = \frac{m}{r} \ln\left(\frac{4+b}{2+b}\right)} \numberthis \label{eq:secondunknown}
        \end{align*}
    \end{itemize}

    We now have two equations with $b$ as our unknown. We shall try to combine the two and solve for $b$:

    If we multiply \eqref{eq:secondunknown} by 2:
    \begin{align*}
        &\left.1 = \frac{m}{r} \ln\left(\frac{4+b}{2+b}\right)\right|\times 2 \\
        &2 = \frac{m}{r} 2 \ln\left(\frac{4+b}{2+b}\right) \\
        2 &= \frac{m}{r} \ln\left(\frac{4+b}{2+b}\right)^2 \\
        &\boxed{\frac{2r}{m} = \ln\left(\frac{4+b}{2+b}\right)^2} \label{eq:thirdunknown} \numberthis
    \end{align*}

    Then if we solve for the $\ln$ in \eqref{eq:firstunknown}:
    \begin{align*}
        &\frac{m}{r} \ln\left(\frac{2+b}{b}\right) = 2 \\
        &\boxed{\frac{2r}{m} = \ln\left(\frac{2+b}{b}\right)} \numberthis \label{eq:fourthunknown}
    \end{align*}

    Then, equating the right hand side members of \eqref{eq:thirdunknown} and \eqref{eq:fourthunknown}:
    \begin{align*}
        \ln\left(\frac{2+b}{b}\right) &= \ln\left(\frac{4+b}{2+b}\right)^2 \\
        \frac{2+b}{b} &= \left(\frac{4+b}{2+b}\right)^2 \\
        \frac{2+b}{b} &= \frac{(4+b)^2}{(2+b)^2} \\
        (2+b)^3 &= b(4+b)^2 \\
        b^3 + 6b^2 + 12b + 8 &= b (b^2 + 8b + 16) \\
        \cancel{b^3} + 6b^2 + 12b + 8 &= \cancel{b^3} + 8b^2 + 16b \\
         2b^2 +  4b - 8 &= 0 \ \bigg| \div 2 \\ 
         b^2 + 2b - 4 &= 0 \\
         \overset{\text{quadratic formula}}{\iff} b_1 = \sqrt{5} - 1 \approx 1.236 &\qquad b_2 = -\sqrt 5 - 1 \approx - 3.236
    \end{align*}

    Since $b$ is the number of hours since it started snowing until the moment at which the snowplow set out to clear the snow (9:00 AM), $b$ must be a positive number so $b >0$ and the second solution $b_2$ is rejected.

    In conclusion, it started snowing $1.236$ hours before 9:00 AM. If we want to be specific:
    \begin{align*}
        0.236 \textnormal{hours is} 0.236 \cdot 60 = 14.16 \textnormal{minutes} \\
        0.16 \textnormal{minutes is} 0.16 \cdot 60 = 9.6 \textnormal{seconds}
    \end{align*}
    so $1.236$ hours equate to $1$ hour, $14$ minutes and $9.6$ seconds
    \[
        9{:}00{:}00 - 1{:}14{:}9.6 = 7{:}45{:}50.4 \textnormal{AM}
    \]
   It started snowing at $7{:}45{:}50.4$ AM.

   \begin{figure}[h!]
       \centering
         \includegraphics[width=0.9\textwidth]{assets/part_1_graph.png}
        \caption{Graph of the height of the snow with an arbitrary snowfall rate and the distance cleared by the snowplow}
   \end{figure}

   We also provide the Python code used to generate the above graph:
   \begin{lstlisting}
    import numpy as np
    import matplotlib.pyplot as plt

    # Times
    t0 = 7.75   # 7:45 AM (snow starts)
    t1 = 9.00   # 9:00 AM (plow starts)
    t_end = 13  # 1:00 PM (end)
    t = np.linspace(t0, t_end, 400)


    r = 1
    h = r * (t - t0)  # h(t0) = 0


    t_11, x_11 = 11, 2
    k = x_11 / np.log((t_11 - t0) / (t1 - t0))

    # Calcualte x(t)
    x = np.zeros_like(t)
    mask = t >= t1
    x[mask] = k * np.log((t[mask] - t0) / (t1 - t0))



    t_13, expected_x_13 = 13, 3
    x_13 = k * np.log((t_13 - t0) / (t1 - t0))
    print(f"Predicted x(13) = {x_13:.2f} km (expected {expected_x_13} km)")




    plt.figure(figsize=(10, 6))
    plt.plot(t, h, label="Snow height  h(t) = r.(t - 7.75)", color='royalblue', linewidth=2)
    plt.plot(t, x, label="Road cleared  x(t)", color='darkorange', linewidth=2)

    # Known cleared points
    plt.scatter([t1, 11, 13], [0, 2, 3], color='red', zorder=5)
    plt.text(11, 2.1, "11 AM (2 km)", color='red', fontsize=10)
    plt.text(13, 3.1, "1 PM (3 km)", color='red', fontsize=10)


    plt.axvline(t0, color='gray', linestyle=':', linewidth=1)
    plt.text(t0 + 0.05, max(h)*0.7, "Snow starts (7:45 AM)", color='gray', fontsize=10, rotation=90, va='center')
    plt.axvline(t1, color='gray', linestyle='--', linewidth=1)
    plt.text(t1 + 0.05, max(h)*0.7, "Plow starts (9:00 AM)", color='gray', fontsize=10, rotation=90, va='center')


    plt.title("Snow Height and Road Cleared vs Time", fontsize=14)
    plt.xlabel("Time (hours)", fontsize=12)
    plt.ylabel("Snow height / Distance cleared [km]", fontsize=12)
    plt.legend()
    plt.grid(True, linestyle='--', alpha=0.6)
    plt.xticks([7.75, 9, 10, 11, 12, 13],
            ["7:45", "9:00", "10:00", "11:00", "12:00", "13:00"])

    plt.show()
   \end{lstlisting}


\setcounter{section}{0}
\newpage
\part{Two Snowplows}
Now we shall move on to the second part of the project, respectively problem E from \cite[pp.~84--85]{Kent_Nagle2018-ig}
\section{Problems}
\begin{framed}
    One day it began to snow exactly at noon at a heavy and steady rate. A snowplow left its garage at 1:00 PM., and another one followed in its tracks at 2:00 PM.
\end{framed}
\begin{figure}[h!]
    \centering
    \includegraphics[width=0.7\textwidth]{assets/two_snowplows_diagram.png}
    \caption{Two snowplows following the same path}
\end{figure}
    \subsection{First subpoint}
        \begin{framed}
            At what time did the second snowplow crash into the first?
        \end{framed}
        We keep the same assumption as part \ref{firstpart}:
        \begin{itemize}
            \item The rate at which a snowplow can clear a road is inversely proportional to the height (or depth) of the snow, and thus to the time elapsed since the road was clear of snow.
        \end{itemize}
    \subsection{Second subpoint}
        \begin{framed}
            Could the crash have been avoided by dispatching the second snowplow at 3:00 PM instead?
        \end{framed}
\section{Solutions}
    \subsection{Time of crash}
        We consider the following notations:
        \begin{itemize}
            \item $t$ - time since noon (12:00 PM) (hours)
            \item $r$ - rate of snowfall (cm/h)
            \item $x(t)$ - the distance that the first snowplow has travelled ($\unit{\kilo\meter}$) at time $t$
            \item $y(t)$ - the distance that the second snowplow has travelled ($\unit{\kilo\meter}$) at time $t$
            \item $h(t)$ - the height of snow ($\unit{\centi\meter}$) at time $t$
        \end{itemize}

        The time of the crash is the moment in time $t$ such that
        \[
            x(t) = y(t)
        \]

        Firstly, we must determine the equations $x(t)$ and $y(t)$.
        
        \hfill \break
        Since the rate at which a snowplow can clear a road is inversely proportional to the height (or depth) of the snow, the product of the speed (\textit{the derivative $x'(t) \textnormal{or} y'(t)$ of the position $x(t) \textnormal{or} y(t)$}) of any of the two snowplows with the height $h(t)$ of the snow is constant, and we shall denote it with $k$:
        \begin{align*}
           & x'(t) \cdot h(t) = k, \quad k \in \mathbb{R} \\
            &y'(t) \cdot h(t) = k
        \end{align*}

        The rate at which it is snowing is steady i.e. constant, thus $h(t)$ is a linear function with slope $r$ and equation
        \begin{gather*}
            h(t) = rt + C, \quad C \in \mathbb{R}
        \end{gather*}

        Since it began snowing at noon (12:00 PM), which we consider our starting time $t = 0$

        \begin{gather*}
            h(0) = 0 \\
            r \cdot 0 + C = 0 \\
            \boxed{C = 0}
        \end{gather*}

        So, $h(t)$ becomes

        \begin{equation*}
            h(t) = rt
        \end{equation*}

        and the proportionality equation for the first snowplow becomes:

        \begin{gather*}
            x'(t) \cdot rt = k \\x'(t) = \frac{k}{rt}
        \end{gather*}

        Let us denote the fraction of the two constants $\frac{k}{r}$ with $A$:

        \begin{gather*}
            x'(t) = \frac{A}{t}
        \end{gather*}

        We're getting closer to finding the equations $x(t)$ and $y(t)$.

        \hfill \break
        Let us integrate $x(t)$:
        \begin{align*}
            \int x'(t) \d t &= \int{\frac{A}{t}} \d t \\
            x(t) &= A \ln |t|+C \\
            t \geq 0 \implies x(t) &= A \ln t + C
        \end{align*}
        
        Let us use our given information:
        \begin{itemize}
            \item The first snowplow leaves the garage at 1:00 PM, 1 hour after noon (12:00 PM), so
                \begin{gather*}
                    x(1) = 0 \\
                    A \ln 1 +C = 0 \\
                    C = 0
                \end{gather*}
                and $x(t)$ becomes
                \begin{equation}
                    x(t) = A \ln t
                    \label{eq:x(t)}
                \end{equation}

            \item The second snowplow left at 2:00 PM.

            \textbf{If the second snowplow has traveled $y(t)$ kilometers at time $t$, then the first snowplow traveled the same distance at time $\tau$, $\tau \in \mathbb{R^+}, \ \tau < t$.}

            \begin{align*}
                y(t) &= x(\tau) \\
                y(t) &= A \ln \tau
            \end{align*}

            Solving for $\tau$, so we can substitute it to have less unknowns in our expressions later on:
            \begin{gather*}
                \ln \tau = \frac{y(t)}{A} \\
                \tau = e^{\frac{y(t)}{A}}
            \end{gather*}

            \textbf{The height of snow that the second snowplow must clear at time $t$ is the snow that has fallen since the first snowplow has passed, time equal to $t - \tau$.} And since the rate of snowfall is constant ($r$), the height of the snow that has fallen during \(t-\tau\) hours is equal to \(r \cdot (t - \tau)\).

            \begin{gather*}
                y'(t) \cdot h(t) = k \\
                y'(t) \cdot r(t-\tau) = k \\
            \end{gather*}
            replacing $\tau$ with \(e^{\frac{y(t)}{A}}\) we obtain
            \begin{gather*}
                y'(t) = \frac{k}{r(t-e^{\frac{y(t)}{A}})}
            \end{gather*}
            and denoting \(\frac{k}{r}\) with \(A\):
            \begin{gather*}
                y'(t) = \frac{A}{t - e^{\frac{y(t)}{A}}}
            \end{gather*}
            which is a \textbf{first-order linear differential equation} with \(y\) as the dependent variable, and \(t\) as the independent variable. We must solve it to obtain the equation of \(y(t)\).
            \begin{gather*}
                \frac{\d y}{\d t} = \frac{A}{t - e^{\frac{y}{A}}} \\
            \end{gather*}
            \cite{Kent_Nagle2018-ig} hints to setting \(t\) as the dependent variable and \(y\) as the independent variable. This way we will obtain \(t(y)\) (\emph{the moment in time $t$ that the second snowplow has cleared $y$ kilometers of snow}).
            \begin{gather*}
                \frac{\mathrm{d}t}{\mathrm{d}y} = \frac{1}{\mathrm{d}y / \mathrm{d}t} = \frac{t - e^{\frac{y}{A}}}{A}
            \end{gather*}
            Since this is a first-order linear differential equation, we shall solve it using the integrating factor. We bring the differential equation into standard form \(\frac{\d t}{\d y} + P(y)t = Q(y)\)
            \begin{gather*}
                \frac{\d t}{\d y} = \frac{t - e^{\frac{y}{A}}}{A} \\
                \frac{\d t}{\d y} = \frac{t}{A} - \frac{e^{\frac{y}{A}}}{A} \\
                \frac{\d t}{\d y} - \frac{1}{A}t = - \frac{e^{\frac{y}{A}}}{A}
            \end{gather*}
            The integrating factor \(\mu(y)\) is equal to:
            \begin{gather*}
                \mu(y) = e^{\int P(y) \d y} \\
                \mu(y) = e^{\int -\frac{1}{A} \d y} \\
                \mu(y) = e^{-\frac{y}{A}}
            \end{gather*}
            Now, we multiply both sides of the differential equation with the integrating factor:
            \begin{gather*}
                \frac{\d t}{\d y} - \frac{1}{A}t = - \frac{e^{\frac{y}{A}}}{A} \Bigg| \times e^{-\frac{y}{A}} \\
                e^{-\frac{y}{A}} \frac{\d t}{\d y} - e^{-\frac{y}{A}} \frac{1}{A} t = \frac{-e^{\frac{y}{A}} \cdot e^{-\frac{y}{A}}}{A} \\
                e^{-\frac{y}{A}} \frac{\d t}{\d y} - e^{-\frac{y}{A}} \frac{1}{A} t = - \frac{1}{A}
            \end{gather*}
            If we write the equation replacing \(\frac{\d t}{\d y}\) with \(t'\), we notice that the left hand side resembles the product derivation rule:
            \begin{gather*}
                \underbrace{e^{-\frac{y}{A}} t' - e^{-\frac{y}{A}} \frac{1}{A} t} = - \frac{1}{A} \\
                e^{-\frac{y}{A}} t' - e^{-\frac{y}{A}} \frac{1}{A} t \equiv (e^{-\frac{y}{A}})t' + (e^{-\frac{y}{A}})'t \equiv (t \cdot e^{-\frac{y}{A}})' \equiv \frac{\d \left(t \cdot e^{-\frac{y}{A}}\right)}{\d y}
            \end{gather*}
            So, we have
            \begin{gather*}
                \frac{\d \left(t \cdot e^{-\frac{y}{A}}\right)}{\d y} = -\frac{1}{A}
            \end{gather*}
            and we integrate both sides with respect to \(y\):
            \begin{gather*}
                \int\frac{\d \left(t \cdot e^{-\frac{y}{A}}\right)} \d y= \int -\frac{1}{A} \d y \\
                \left(t \cdot e^{-\frac{y}{A}}\right) = -\frac{y}{A} + C, \quad C \in \mathbb{R}
            \end{gather*}
            Solving for \(t\) we get the formula for \(t(y)\):
            \begin{gather*}
                t(y) = \frac{1}{e^{-\frac{y}{A}}}\left(C-\frac{y}{A}\right) = e^\frac{y}{A}\left(C - \frac{y}{A}\right) \numberthis \label{eq:t(y)_initial}
            \end{gather*}
            At 2:00 PM, $t = 2$ hours after 12:00, the second snowplow has cleared $y = 0$ kilometers of snow, thus
            \begin{gather*}
                t(0) = 2 \\
                e^{\frac{0}{A}}\left(C - \frac{0}{A}\right) = 2 \\
                e^0 \left(C - 0\right) = 2 \\
                C = 2
            \end{gather*}
            So, finally, we have the general formula for \(t(y)\):
            \begin{gather*}
                t(y) = e^\frac{y}{A}\left(2-\frac{y}{A}\right) \numberthis \label{eq:t(y)}
            \end{gather*}
        \end{itemize}

        Now that we have the equations \eqref{eq:x(t)}, \eqref{eq:t(y)} for \(x(t)\) and \(t(y)\) respectively, we shall combine them.

        \hfill \break
        As we mentioned, the snowplows collide when
        \begin{equation*}
            y = x = A \ln t
        \end{equation*}
        By substituting $y$ in equation \eqref{eq:t(y)} and solving for \(t\), we obtain
        \begin{gather*}
            t = e^\frac{y}{A}\left(2-\frac{y}{A}\right) \\
            t = e^\frac{A \ln t}{A}\left(2-\frac{A \ln t}{A}\right) \\
            t = e^{\ln t}\left(2 - \ln t\right) \\
            t = t (2-\ln t) \Bigg | \div t, \ \ t > 0 \\ 
            2-\ln t = 1 \\
            \ln t = 1 \\
            e^{\ln t} = e^1 \\
            \boxed{t = e}
        \end{gather*}

        In conclusion, the two snowplows crashed into one another \(e \approx 2.718\) hours after 12:00 PM (noon). If we want to be specific:
        \begin{gather*}
            0.718 \textnormal{hours is} 0.718 \cdot 60 = 43.08 \textnormal{minutes} \\
            0.08 \textnormal{minutes is} 0.08 \cdot 60 = 4.8 \textnormal{seconds}
        \end{gather*}

        So \(e\) hours equate to 2 hours, 43 minutes and 4.8 seconds
        \[
            12{:}00{:}00 + 2{:}43{:}4.8 = 14{:}43{:}4.8 = 2{:}43{:}4.8 \textnormal{PM}
        \]
        The second snowplow crashed into the first at \(2{:}43{:}4.8 \text{ PM}\)\footnote{Note that because of the irrationality of Euler's number \(e\), the number of seconds might be slightly inaccurate.}.
        
    \subsection{Preventing the crash}
        Could the crash be prevented by dispatching the second snowplow at 3:00 PM?

        \hfill \break
        Formulas remain the same, except for the constant \(C\) in the initial equation of \(t(y)\) (equation \eqref{eq:t(y)_initial}).
        The second snowplow clears \(0\) kilometers of snow at \(t = 3\) hours after noon at which it gets dispatched, so
        \begin{gather*}
            t(0) = 3 \\
            e^{\frac{0}{A}}\left(C - \frac{0}{A}\right) = 3 \\
            C = 3
        \end{gather*}

        Using the same method as before to find the value of \(t\) for when the two snowplows crash, we get
        \begin{gather*}
            t = e^{\frac{y}{A}}\left(3-\frac{y}{A}\right) \\
            t = e^\frac{A \ln t}{A}\left(3 - \frac{A \ln t}{A}\right) \\
            t = t(3 - \ln t) \\
            1 = 3 - \ln t \\
            t = e^2 \approx 7.389
        \end{gather*}

        In conclusion, if the first snowplow does not finish its course (or at least deviate from it) by $e^2$ hours after noon, then a crash would still happen.

        \hfill \break
        As a matter of fact, for an arbitrary number $T, \ T > 1$ of hours after noon at which the second snowplow would begin its course, the crash will happen
        \[ t = e^{T-1} \]
        hours after noon. Meaning, as long as there is no change in speed, rate of snowfall or the trail that the snowplows follow, \emph{there will always be a crash}.

        \hfill \break
        Finally, we provide a graph of the height of the snow with an arbitrary snowfall rate and the distance cleared by the two snowplows. The graph was made using the same Python code as in part \ref{firstpart}, with minor modifications to account for the second snowplow.
        \begin{figure}[h!]
            \centering
            \includegraphics[width=0.9\textwidth]{assets/two_snowplows_graph.png}
            \caption{Graph of the height of the snow with an arbitrary snowfall rate and the distance cleared by the two snowplows}
        \end{figure}
    

\newpage \printbibliography

\end{document}
