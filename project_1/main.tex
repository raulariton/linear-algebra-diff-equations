\documentclass[a4paper,12pt]{article}
\usepackage[english]{babel}
\usepackage[utf8]{inputenc}
\usepackage[T1]{fontenc}

\usepackage{graphicx}

\usepackage{lmodern}

% math symbols and fonts
\usepackage{amsmath}
\usepackage{amssymb}
\usepackage{mathrsfs}

% d for derivatives
\usepackage{fixdif}

% SI units
\usepackage{siunitx}

% framed environment, might use in problem introduction
\usepackage{framed}

% bibtex setup
\usepackage[backend=biber,
            style=ieee,
            ]{biblatex}
\addbibresource{bibliography.bib}

\usepackage{cancel}

% context-sensitive quotes package (biblatex suggests using it, idk what it does exactly)
\usepackage{csquotes}

% hyperref package; convention to always add as last package
\usepackage{hyperref}

% adding text spacing in math mode
\renewcommand*{\textnormal}[1]{\text{ #1 }}

% removing highlighting of hyperref links
\hypersetup{pdfborder={0 0 0}}

% remove indentation
\setlength{\parindent}{0pt}

\newcommand\numberthis{\addtocounter{equation}{1}\tag{\theequation}}

\begin{document}

\begin{titlepage}
	\vspace*{-3cm}
	\centering
		\includegraphics[width=2cm]{assets/uvt_logo_en.png}

	{\scshape\LARGE West University of Timișoara\par}

	\vspace{1cm}

	{\large Faculty of Computer Science}

	\vspace{.1\textheight}
	{ \LARGE \scshape Project 1: \\ Snowplow Problem \par}

		\vspace{.1\linewidth}

			{\emph{Authors:} \\
			Maria-Miruna \textsc{Mesărășoiu} (\textit{Group 1}) \\
            Silviu-Ștefan \textsc{Mitrea} (\textit{Group 1}) \\
            Raul-Andrei \textsc{Ariton} (\textit{Group 2})    
            }

			\vfill

			\rule{.4\textwidth}{.4pt}

	{\large \today\par
	Academic Year 2025-2026\par}
\end{titlepage}

\tableofcontents \newpage

\part{The Snowplow Problem} \label{firstpart}
\section{Problem}
    Firstly, we shall introduce the problem, as described in \cite[pp.~84--85]{Kent_Nagle2018-ig}. Note that for practicality, we have decided to use kilometers instead of miles.

    \begin{framed}
        \noindent One morning it began to snow very hard and continued snowing steadily throughout the day. A snowplow set out at 9:00 AM to clear a road, clearing 2 \unit{\kilo\meter} by 11:00 AM. and an additional kilometer by 1:00 PM. At what time did it start snowing?
    \end{framed}
    % image, tikZ plot ? :)
    
    The source also mentions the following assumptions:
    \begin{itemize}
        \item It is snowing at a constant rate.
        \item The rate at which the snowplow can clear a road is inversely proportional to the height (or depth) of the snow.
    \end{itemize}

\section{Solution}
    We shall now begin the attempt at solving the problem. 

    \noindent Let us consider the following notations:
    \begin{itemize}
        \item $t$ - time since 9:00 AM (hours)
        \item $x(t)$ - the distance that the snowplow has travelled ($\unit{\kilo\meter}$) at time $t$
        \item $h(t)$ - the height of the snow ($\unit{\centi\meter}$) at time $t$
    \end{itemize}
    Additionally:
    \begin{itemize}
        \item b - the number of hours before 9:00 AM when it started snowing
        \item r - rate of snowfall ($\unit{\centi\meter / h}$)
    \end{itemize}

    Since it is snowing at a constant rate, the height of the snow $h$ is increasing at a constant rate. Mathematically this means that the derivative of $h$ is constant, and equal to $r$, $r \in \mathbb{R}$.

    \begin{align*}
        h'(t) = \frac{\d h}{\d t} = r
    \end{align*}

    Thus $h$ is a linear function with slope $r$ and with the equation
    \begin{align*}
        h(t) = r t+ C, \quad C \in \mathbb{R}
    \end{align*}
    That's a lot of letters and unknowns. Let us attempt to substitute the constant $C$ in the equation of $h(t)$ with something more relevant to our problem.
    We consider $t = 0$ the moment that the snowplow sets out to clear the road. Similarly, $t = 0-b$ is the moment that it starts snowing, and the height of the snow is equal to $0$.
    \begin{align*}
        &h(0-b) = r (0-b) + C \\
        &0 = -rb + C \\
        &\boxed{C = rb}
    \end{align*}
    And now we can substitute $C$ in the equation of $h$:
    \begin{align*}
        h(t) = rt + rb \\
        h(t) = r(t+b)
    \end{align*}
    We know that the rate at which the snowplow moves (clears a road) is inversely proportional to the height of the snow:
    \begin{align*}
        x'(t) = \frac{\d x}{\d t} \propto \frac{1}{h(t)}
    \end{align*}
    Then the product of the two is constant:
    \begin{align*}
        x'(t) h(t) &= m, \quad m \in \mathbb{R} \\
        \iff x'(t) &= \frac{m}{h(t)} \\
        \iff x'(t) &= \frac{m}{r(t+b)}
    \end{align*}
    Since $m$ and $r$ are constants, we can rewrite the equation as follows, separating the constants from the variables and unknowns:
    \begin{equation*}
        x'(t) = \frac{m}{r} \frac{1}{(t+b)}
    \end{equation*}

    % ^ this is a differential equation

    Since we have information about the kilometers the snowplow clears after a known number of hours, let us attempt to find the equation of $x(t)$. For this, we will integrate $x'(t)$:
    \begin{align*}
        x(t) &= \int x'(t) \d t \\ 
        &= \int \frac{m}{r} \frac{1}{(t+b)} \d t \\
        &= \frac{m}{r} \int \frac{1}{(t+b)} \d t \\
        &= \frac{m}{r} \ln(t+b) + C_2, \quad C_2 \in \mathbb{R}
    \end{align*}
    Note that the product $\frac{m}{r}C_2$ is just a product of constants and thus unnecessary to write.
    Now using the equation for $x(t)$, we substitute using known information:
    \begin{itemize}
        \item The snowplow cleared $2 \unit{\kilo\meter}$ between 9:00 AM and 11:00 AM (the span of two hours):
        \begin{align*}
            x(2) - x(0) &= \left.\frac{m}{r}\ln(t+b)\right|^2_0 \\
            2 &= \left(\frac{m}{r} \ln(2+b)\right) - \left(\frac{m}{r}\ln(0+b)\right) \\
            2 &= \frac{m}{r} \left(\ln(2+b) - \ln(0+b)\right) \\
            &\boxed{2 = \frac{m}{r} \ln\left(\frac{2+b}{b}\right)} \numberthis \label{eq:firstunknown}
        \end{align*}
        \item The snowplow cleared $1 \unit{\kilo\meter}$ between 11:00 AM (2 hours after it set out) and 1:00 PM (4 hours after it set out):
        \begin{align*}
            x(4) - x(2) &= \left.\frac{m}{r}\ln(t+b)\right|^4_2 \\
            1 &= \left(\frac{m}{r} \ln(4+b)\right) - \left(\frac{m}{r}\ln(2+b)\right) \\
            1 &= \frac{m}{r} \left(\ln(4+b) - \ln(2+b)\right) \\
            &\boxed{1 = \frac{m}{r} \ln\left(\frac{4+b}{2+b}\right)} \numberthis \label{eq:secondunknown}
        \end{align*}
    \end{itemize}

    We now have two equations with $b$ as our unknown. We shall now try to combine the two and solve for $b$:

    If we multiply \eqref{eq:secondunknown} by 2:
    \begin{align*}
        &\left.1 = \frac{m}{r} \ln\left(\frac{4+b}{2+b}\right)\right|\times 2 \\
        &2 = \frac{m}{r} 2 \ln\left(\frac{4+b}{2+b}\right) \\
        2 &= \frac{m}{r} \ln\left(\frac{4+b}{2+b}\right)^2 \\
        &\boxed{\frac{2r}{m} = \ln\left(\frac{4+b}{2+b}\right)^2} \label{eq:thirdunknown} \numberthis
    \end{align*}

    Then if we solve for the $\ln$ in \eqref{eq:firstunknown}:
    \begin{align*}
        &\frac{m}{r} \ln\left(\frac{2+b}{b}\right) = 2 \\
        &\boxed{\frac{2r}{m} = \ln\left(\frac{2+b}{b}\right)} \numberthis \label{eq:fourthunknown}
    \end{align*}

    Then, equating the right hand side members of \eqref{eq:thirdunknown} and \eqref{eq:fourthunknown}:
    \begin{align*}
        \ln\left(\frac{2+b}{b}\right) &= \ln\left(\frac{4+b}{2+b}\right)^2 \\
        \frac{2+b}{b} &= \left(\frac{4+b}{2+b}\right)^2 \\
        \frac{2+b}{b} &= \frac{(4+b)^2}{(2+b)^2} \\
        (2+b)^3 &= b(4+b)^2 \\
        b^3 + 6b^2 + 12b + 8 &= b (b^2 + 8b + 16) \\
        \cancel{b^3} + 6b^2 + 12b + 8 &= \cancel{b^3} + 8b^2 + 16b \\
         2b^2 +  4b - 8 &= 0 \ \bigg| \div 2 \\ 
         b^2 + 2b - 4 &= 0 \\
         \overset{\text{quadratic formula}}{\iff} b_1 = \sqrt{5} - 1 \approx 1.236 &\qquad b_2 = -\sqrt 5 - 1 \approx - 3.236
    \end{align*}

    Since $b$ is the number of hours before 9:00 AM, the moment at which the snowplow set out to clear the snow, $b$ must be a positive number so $b >0$ and the second solution $b_2$ is rejected.

    In conclusion, it started snowing $1.236$ hours before 9:00 AM. If we want to be specific:
    \begin{align*}
        0.236 \textnormal{hours is} 0.236 \cdot 60 = 14.16 \textnormal{minutes} \\
        0.16 \textnormal{minutes is} 0.16 \cdot 60 = 9.6 \textnormal{seconds}
    \end{align*}
    so $1.236$ hours equate to $1$ hour, $14$ minutes and $9.6$ seconds
    \[
        9{:}00{:}00 - 1{:}14{:}9.6 = 7{:}45{:}50.4 \textnormal{AM}
    \]
   It started snowing at $7{:}45{:}50.4$ AM.

\part{Two Snowplows}
Now we shall move on to the second part of the project, respectively problem D from \cite[pp.~84--85]{Kent_Nagle2018-ig}
\section{Problems}
\begin{framed}
    One day it began to snow exactly at noon at a heavy and steady rate. A snowplow left its garage at 1:00 PM., and another one followed in its tracks at 2:00 PM.
\end{framed}
    \subsection{First subpoint}
        \begin{framed}
            At what time did the second snowplow crash into the first?
        \end{framed}
        We keep the same assumption as part \ref{firstpart}:
        \begin{itemize}
            \item The rate at which a snowplow can clear a road is inversely proportional to the height (or depth) of the snow, and thus to the time elapsed since the road was clear of snow.
        \end{itemize}
    \subsection{Second subpoint}
        \begin{framed}
            Could the crash have been avoided by dispatching the second snowplow at 3:00 PM instead?
        \end{framed}
\section{Solutions}
    \subsection{Time of crash}
        We consider the following notations:
        \begin{itemize}
            \item $t$ - time since noon (12:00 PM) (hours)
            \item $r$ - rate of snowfall (cm/h)
            \item $x(t)$ - the distance that the first snowplow has travelled ($\unit{\kilo\meter}$) at time $t$
            \item $y(t)$ - the distance that the second snowplow has travelled ($\unit{\kilo\meter}$) at time $t$
            \item $h(t)$ - the height of snow ($\unit{\centi\meter}$) at time $t$
        \end{itemize}

        The time of the crash is the moment in time $t$ such that
        \[
            x(t) = y(t)
        \]

        Firstly, we must determine the equations $x(t)$ and $y(t)$.
        
        \hfill \break
        Since the rate at which a snowplow can clear a road is inversely proportional to the height (or depth) of the snow, the product of the speed (\textit{the derivative $x'(t) \textnormal{or} y'(t)$ of the position $x(t) \textnormal{or} y(t)$}) of any of the two snowplows with the height $h(t)$ of the snow is constant, and we shall denote it with $k$:
        \begin{align*}
           & x'(t) \cdot h(t) = k, \quad k \in \mathbb{R} \\
            &y'(t) \cdot h(t) = k
        \end{align*}

        The rate at which it is snowing is steady i.e. constant, thus $h(t)$ is a linear function with slope $r$ and equation
        \begin{gather*}
            h(t) = rt + C, \quad C \in \mathbb{R}
        \end{gather*}

        Since it began snowing at noon (12:00 PM), which we consider our starting time $t = 0$

        \begin{gather*}
            h(0) = 0 \\
            r \cdot 0 + C = 0 \\
            \boxed{C = 0}
        \end{gather*}

        So, $h(t)$ becomes

        \begin{equation*}
            h(t) = rt
        \end{equation*}

        and the proportionality equation for the first snowplow becomes:

        \begin{gather*}
            x'(t) \cdot rt = k \\x'(t) = \frac{k}{rt}
        \end{gather*}

        Let us denote the fraction of the two constants $\frac{k}{r}$ with $A$:

        \begin{gather*}
            x'(t) = \frac{A}{t}
        \end{gather*}

        We're getting closer to finding the equations $x(t)$ and $y(t)$.

        \hfill \break
        Let us integrate $x(t)$:
        \begin{align*}
            \int x'(t) \d t &= \int{\frac{A}{t}} \d t \\
            x(t) &= A \ln |t|+C \\
            t \geq 0 \implies x(t) &= A \ln t + C
        \end{align*}
        
        Let us use our given information:
        \begin{itemize}
            \item The first snowplow leaves the garage at 1:00 PM, 1 hour after noon (12:00 PM), so
                \begin{gather*}
                    x(1) = 0 \\
                    A \ln 1 +C = 0 \\
                    C = 0
                \end{gather*}
                and $x(t)$ becomes
                \begin{equation}
                    x(t) = A \ln t
                    \label{eq:x(t)}
                \end{equation}

            \item The second snowplow left at 2:00 PM.

            If the second snowplow has cleared $y(t)$ kilometers of snow at time $t$, then the first snowplow cleared the same amount of snow at time $\tau$\footnote{ We use $\tau$ purely for notation purposes, its true value is out of the scope for this exercise.}, $\tau \in \mathbb{R^+}, \ \tau < t$.

            \begin{align*}
                y(t) &= x(\tau) \\
                y(t) &= A \ln \tau
            \end{align*}

            Solving for $\tau$, so we can substitute it to have less unknowns in our expressions:

            \begin{align*}
                \ln \tau &= \frac{y(t)}{A} \\
                \tau &= e^{\frac{y(t)}{A}}
            \end{align*}

            The height of snow that the second snowplow must clear at time $t$ is the snow that has fallen since the first snowplow has passed, time equal to $t - \tau$. And since the rate of snowfall is constant ($r$), the height of the snow that has fallen during \(t-\tau\) hours is equal to \(r \cdot (t - \tau)\).

            \begin{align*}
                y'(t) \cdot h(t) &= k \\
                y'(t) \cdot r(t-\tau) &= k \\
                \intertext{replacing $\tau$ with \(e^{\frac{y(t)}{A}}\) we obtain}
                y'(t) &= \frac{k}{r(t-e^{\frac{y(t)}{A}})}
                \intertext{and denoting \(\frac{k}{r}\) with \(A\):}
                y'(t) &= \frac{A}{t - e^{\frac{y(t)}{A}}}
                \intertext{which is a \textbf{first-order linear differential equation} with \(y\) as the dependent variable, and \(t\) as the independent variable. We must solve it to obtain the equation of \(y(x)\).}
                \frac{\d y}{\d t} &= \frac{A}{t - e^{\frac{y}{A}}} \\
                \intertext{\cite{Kent_Nagle2018-ig} hints to setting \(t\) as the dependent variable and \(y\) as the independent variable. This way we will obtain \(t(y)\) (\emph{the moment in time $t$ that the second snowplow has cleared $y$ kilometers of snow}).}
                \frac{\mathrm{d}t}{\mathrm{d}y} &= \frac{1}{\mathrm{d}y / \mathrm{d}t} = \frac{t - e^{\frac{y}{A}}}{A}
                \intertext{Since this is a first-order linear differential equation, we shall solve it using the integrating factor. We bring the differential equation into standard form \(\frac{\d t}{\d y} + P(y)t = Q(y)\)}
                \frac{\d t}{\d y} &= \frac{t - e^{\frac{y}{A}}}{A} \\
                \frac{\d t}{\d y} &= \frac{t}{A} - \frac{e^{\frac{y}{A}}}{A} \\
                \frac{\d t}{\d y} - \frac{1}{A}t &= - \frac{e^{\frac{y}{A}}}{A}
                \intertext{The integrating factor \(\mu(y)\) is equal to:}
                \mu(y) &= e^{\int P(y) \d y} \\
                \mu(y) &= e^{\int -\frac{1}{A} \d y} \\
                \mu(y) &= e^{-\frac{y}{A}}
                \intertext{Now, we multiply both sides of the differential equation with the integrating factor:}
                \frac{\d t}{\d y} - \frac{1}{A}t &= - \frac{e^{\frac{y}{A}}}{A} \Bigg| \times e^{-\frac{y}{A}} \\
                e^{-\frac{y}{A}} \frac{\d t}{\d y} - e^{-\frac{y}{A}} \frac{1}{A} t &= \frac{-e^{\frac{y}{A}} \cdot e^{-\frac{y}{A}}}{A} \\
                e^{-\frac{y}{A}} \frac{\d t}{\d y} - e^{-\frac{y}{A}} \frac{1}{A} t &= - \frac{1}{A}
                \intertext{If we write the equation replacing \(\frac{\d t}{\d y}\) with \(t'\), we notice that the left hand side resembles the product derivation rule:}
                \underbrace{e^{-\frac{y}{A}} t' - e^{-\frac{y}{A}} \frac{1}{A} t} &= - \frac{1}{A} \\
                e^{-\frac{y}{A}} t' - e^{-\frac{y}{A}} \frac{1}{A} t \equiv (e^{-\frac{y}{A}})t' &+ (e^{-\frac{y}{A}})'t \equiv (t \cdot e^{-\frac{y}{A}})' \equiv \frac{\d \left(t \cdot e^{-\frac{y}{A}}\right)}{\d y}
                \intertext{So, we have}
                \frac{\d \left(t \cdot e^{-\frac{y}{A}}\right)}{\d y} &= -\frac{1}{A}
                \intertext{and we integrate both sides with respect to \(y\):}
                \int\frac{\d \left(t \cdot e^{-\frac{y}{A}}\right)} \d y &= \int -\frac{1}{A} \d y \\
                \left(t \cdot e^{-\frac{y}{A}}\right) &= -\frac{y}{A} + C, \quad C \in \mathbb{R}
                \intertext{Solving for \(t\) we get the formula for \(t(y)\)}
                t(y) = \frac{1}{e^{-\frac{y}{A}}}\left(C-\frac{y}{A}\right) &= e^\frac{y}{A}\left(C - \frac{y}{A}\right) \numberthis \label{eq:t(y)_initial}
                \intertext{At 2:00 PM, $t = 2$ hours after 12:00, the second snowplow has cleared $y = 0$ kilometers of snow, thus}
                t(0) &= 2 \\
                e^{\frac{0}{A}}\left(C - \frac{0}{A}\right) &= 2 \\
                e^0 \left(C - 0\right) &= 2 \\
                C &= 2
                \intertext{So, finally, we have the formula for \(t(y)\)}
                t(y) &= e^\frac{y}{A}\left(2-\frac{y}{A}\right) \numberthis \label{eq:t(y)}
            \end{align*}
        \end{itemize}

        Now that we have the equations \eqref{eq:x(t)}, \eqref{eq:t(y)} for \(x(t)\) and \(t(y)\) respectively, we shall combine them.

        \hfill \break
        As we mentioned, the snowplows collide when
        \begin{equation*}
            y = x = A \ln t
        \end{equation*}
        By substituting $y$ in equation \eqref{eq:t(y)} and solving for \(t\), we obtain
        \begin{gather*}
            t = e^\frac{y}{A}\left(2-\frac{y}{A}\right) \\
            t = e^\frac{A \ln t}{A}\left(2-\frac{A \ln t}{A}\right) \\
            t = e^{\ln t}\left(2 - \ln t\right) \\
            t = t (2-\ln t) \Bigg | \div t, \ \ t > 0 \\ 
            2-\ln t = 1 \\
            \ln t = 1 \\
            e^{\ln t} = e^1 \\
            \boxed{t = e}
        \end{gather*}

        In conclusion, the two snowplows crashed into one another \(e \approx 2.718\) hours after 12:00 PM (noon). If we want to be specific:
        \begin{gather*}
            0.718 \textnormal{hours is} 0.718 \cdot 60 = 43.08 \textnormal{minutes} \\
            0.08 \textnormal{minutes is} 0.08 \cdot 60 = 4.8 \textnormal{seconds}
        \end{gather*}

        So \(e\) hours equate to 2 hours, 43 minutes and 4.8 seconds
        \[
            12{:}00{:}00 + 2{:}43{:}4.8 = 14{:}43{:}4.8 = 2{:}43{:}4.8 \textnormal{PM}
        \]
        The second snowplow crashed into the first at \(2{:}43{:}4.8 \text{ PM}\)\footnote{Note that because of the irrationality of Euler's number \(e\), the number of seconds might be slightly inaccurate.}.
        
    \subsection{Preventing the crash}
        Could the crash be prevented by dispatching the second snowplow at 3:00 PM?

        \hfill \break
        Formulas remain the same, except for the constant \(C\) in the initial equation of \(t(y)\) (equation \eqref{eq:t(y)_initial}).
        The second snowplow clears \(0\) kilometers of snow at  $t$ hours after noon at which it gets dispatched, so
        \begin{gather*}
            t(0) = 3 \\
            e^{\frac{0}{A}}\left(C - \frac{0}{A}\right) = 3 \\
            C = 3
        \end{gather*}

        Using the same method as before to find the value of \(t\) for when the two snowplows crash, we get
        \begin{gather*}
            t = e^{\frac{y}{A}}\left(3-\frac{y}{A}\right) \\
            t = e^\frac{A \ln t}{A}\left(3 - \frac{A \ln t}{A}\right) \\
            t = t(3 - \ln t) \\
            1 = 3 - \ln t \\
            t = e^2 \approx 5.436
        \end{gather*}

        In conclusion, if the first snowplow does not finish its course (or at least deviate from it) by $e^2$ hours after noon, then a crash would still happen.

        \hfill \break
        As a matter of fact, for an arbitrary number $T, \ T > 1$ of hours after noon at which the second snowplow would begin its course, the crash will happen
        \[ t = e^{T-1} \]
        hours after noon. Meaning, as long as there is no change in speed, rate of snowfall or the trail that the snowplows follow, \emph{there will always be a crash}.
    

\newpage \printbibliography

\end{document}